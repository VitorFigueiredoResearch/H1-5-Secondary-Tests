\documentclass[11pt,a4paper]{article}


\usepackage{fontspec}
\usepackage{unicode-math}
\setmainfont{Latin Modern Roman}
\setsansfont{Latin Modern Sans}
\setmonofont{Latin Modern Mono}
\setmathfont{latinmodern-math.otf}
\usepackage[a4paper, margin=1in]{geometry}
\usepackage{amsmath}
\usepackage{microtype}
\usepackage{xurl}
\setlength{\emergencystretch}{3em}
\usepackage{graphicx}
\usepackage{booktabs}
\usepackage{caption}
\usepackage{subcaption}
\usepackage{float}
\usepackage[numbers,sort&compress]{natbib}
\bibliographystyle{unsrt} 
\usepackage{hyperref}
\hypersetup{
	colorlinks  = true,
	linkcolor   = blue,
	citecolor   = blue,
	urlcolor    = blue,
	pdfauthor   = {Vítor Manuel Franco Figueiredo},
	pdftitle    = {H1.5: Diagnostic Tests of a Frozen Nonlocal Baryonic Model Against the BTFR and RAR},
	pdfsubject  = {Galaxy Dynamics, Nonlocal Baryonic Model, BTFR, RAR}
}
\usepackage{url}
\urlstyle{tt}
\usepackage{setspace}
\onehalfspacing
\setlength{\parindent}{0pt}
\setlength{\parskip}{6pt}
\newcommand{\Phieff}{\Phi_{\rm eff}}
\newcommand{\Phib}{\Phi_{\rm b}}
\newcommand{\gobs}{g_{\rm obs}}
\newcommand{\gbar}{g_{\rm bar}}
\newcommand{\gdag}{g^{\dagger}}
\newcommand{\Vflat}{V_{\rm flat}}
\newcommand{\Mbar}{M_{\rm b}}
\newcommand{\rfrac}{r_{\rm frac}}
\newcommand{\sparc}{\textsc{sparc}}
\newcommand{\zenodo}{\href{https://doi.org/10.5281/zenodo.18321595}{Zenodo DOI: 10.5281/zenodo.18321595}}
\newcommand{\github}{\href{https://github.com/VitorFigueiredoResearch/H1-5-Secondary-Tests}{GitHub: H1-5-Secondary-Tests}}
\title{
	\textbf{H1.5: Diagnostic Tests of a Frozen Nonlocal Baryonic Model\\
		Against the BTFR and RAR}
}

\author{
	Vítor Manuel Franco Figueiredo\\[4pt]
	\small Independent Researcher, Portugal\\[2pt]
	\small ORCID: \href{https://orcid.org/0009-0004-7358-4622}{0009-0004-7358-4622}
}

\date{January 31, 2026}

\begin{document}
\maketitle

\begin{center}
	\small
	\textit{Data Availability:}
	The full analysis pipeline, diagnostic tables, and figure-generation scripts
	are archived at \zenodo\\ and \github.
	All figures are reproducible without manual intervention.
\end{center}

\vspace{6pt}
\hrule
\vspace{10pt}
\begin{abstract}
We present H1.5, a set of secondary empirical diagnostics applied to the frozen H1
phenomenological model of disk galaxy dynamics. The H1 model encodes a nonlocal
baryon-coupled dynamical response and was previously tested against galaxy rotation
curves without parameter tuning. In this work, the H1 outputs are treated as fixed
inputs and are confronted with two of the strongest empirical scaling relations in
disk galaxies: the baryonic Tully--Fisher relation (BTFR) and the radial acceleration
relation (RAR).

Using the full \sparc\ sample of 175 galaxies, we construct diagnostic tests that
involve no refitting, optimization, or adaptive modification of the model. The BTFR
diagnostic shows that the frozen H1 outputs preserve the existence and overall
structure of the global baryonic mass--velocity relation. In contrast, the RAR
diagnostic reveals substantial scatter relative to the observed empirical relation,
indicating that the fixed H1 construction does not enforce a universal local
acceleration correspondence.

A decomposition of the RAR scatter by radial regime demonstrates that this deviation
is structured rather than stochastic: the largest discrepancies occur in the inner
regions of galaxies, while the outer disk exhibits systematically reduced scatter.
Independent single-galaxy reconstructions confirm numerical consistency at machine
precision, ruling out circularity or post-processing artifacts.

Taken together, these results delineate the empirical domain of validity of the
frozen H1 model. H1.5 does not propose modifications or extensions but instead
provides a transparent diagnostic baseline against which future dynamical frameworks
can be evaluated.
\end{abstract}


\section{Introduction}

The discrepancy between observed galaxy rotation curves and the predictions of Newtonian gravity applied to luminous matter has remained a central problem in astrophysics for over four decades \citep{Rubin1980}. In the standard cosmological paradigm, this discrepancy is attributed to the presence of non-baryonic dark matter halos surrounding galaxies, commonly modeled using universal density profiles such as the Navarro--Frenk--White (NFW) profile \citep{NFW1997}. While this framework successfully reproduces large-scale structure formation, it requires galaxy-by-galaxy halo tuning to match detailed rotation curve data.

Beyond individual rotation curves, disk galaxies exhibit remarkably tight empirical scaling relations that pose additional constraints on theoretical models. Chief among these are the Baryonic Tully--Fisher Relation (BTFR), which links total baryonic mass to asymptotic rotation velocity \citep{McGaugh2000btfr}, and the Radial Acceleration Relation (RAR), which connects the observed centripetal acceleration to that predicted from baryonic mass distributions \citep{McGaugh2016rar, Lelli2017rar}. These relations display low intrinsic scatter and regularity across a wide range of galaxy masses and morphologies, making them stringent benchmarks for any successful theory of galactic dynamics.

Modified gravity and modified inertia approaches, most notably Modified Newtonian Dynamics (MOND), naturally reproduce both the BTFR and the RAR with minimal parameter freedom \citep{Milgrom1983}. Relativistic extensions of MOND have been developed \citep{Bekenstein1984, Skordis2021}, while alternative phenomenological frameworks—including emergent gravity and dark-sector interactions—have also been proposed \citep{Verlinde2017, Berezhiani2015}. Despite these advances, a persistent challenge remains the construction of models that are both predictive and systematically testable without invoking adaptive tuning across individual galaxies.

In this work, we present pre-registered diagnostic tests of a previously introduced phenomenological model, hereafter referred to as H1, which constructs the effective gravitational potential via a non-local convolution of the observed baryonic mass distribution. Crucially, the parameters of H1 are frozen at values determined in prior work and are not refit in the present analysis. This design isolates predictive performance from fitting flexibility, allowing the model to be evaluated strictly on its ability to reproduce independent observational benchmarks. Using the full SPARC database of 175 galaxies \citep{Lelli2016sparc}, we assess the performance of H1 against both the BTFR and the RAR without per-galaxy parameter adjustments.

The paper is organized as follows. Section~2 describes the SPARC dataset and the frozen numerical pipeline used throughout the analysis. Section~3 presents the diagnostic test of H1 against the BTFR. Section~4 examines the RAR, including a radial decomposition of residuals. Section~5 discusses the implications of the results and identifies structured failure modes inherent to single-scale non-local kernels. Section~6 summarizes the conclusions and outlines directions for future work.


\section{Data and Frozen Pipeline}
This work uses the \sparc\ (Spitzer Photometry and Accurate Rotation Curves)
database, which provides high-quality rotation curves and baryonic mass models
for disk galaxies across a wide range of masses, morphologies, and surface
brightnesses \cite{Lelli2016sparc, Lelli2017rar}.


The diagnostic analyses presented in this paper do not involve any new fitting, tuning, or modification of the underlying dynamical model. Instead, they operate exclusively on a fixed set of precomputed rotation curve outputs generated by the H1 model. These outputs were produced prior to the present study and are treated here as fixed inputs.

For each galaxy, the frozen H1 outputs include the total modeled rotation curve and its baryonic decomposition as a function of radius. All quantities used in the BTFR and RAR diagnostics are derived directly from these stored results. No model parameters were altered, re-optimized, or re-estimated at any stage of the H1.5 analysis.

The H1 numerical pipeline is explicitly frozen: the convolution kernel, grid resolution, normalization procedures, and fitting logic are identical to those used in the archived H1 release. The present work therefore constitutes a purely diagnostic extension, designed to test how the fixed H1 outputs behave when confronted with independent empirical scaling relations.

Reproducibility is a core requirement of this study. The complete frozen H1 outputs, together with the analysis scripts used to construct the BTFR and RAR diagnostics, are publicly available through a versioned GitHub repository (\url{https://github.com/VitorFigueiredoResearch/H1-5-Secondary-Tests}) and an archived Zenodo record associated with the H1 model. All figures and numerical values reported in this paper can be regenerated directly from these materials without manual intervention.

We emphasize that the results presented below should be interpreted strictly as properties of the frozen H1 model. Any deviations from empirical relations reflect the behavior of that fixed construction and are not the result of post hoc adjustments or adaptive corrections introduced in the present analysis.
All diagnostic analyses operate on precomputed rotation curve outputs generated by the H1 model, which is treated here as a fixed and immutable input \citep{H1zenodo}.
All numerical computations were performed using double-precision floating point arithmetic, and the analysis scripts were executed under a fixed software environment to ensure reproducibility.


\section{Baryonic Tully--Fisher Relation Diagnostic}

The Baryonic Tully--Fisher Relation (BTFR) provides a global empirical connection
between the total baryonic mass of a disk galaxy and its characteristic rotational
velocity. As a diagnostic test, the BTFR probes whether a dynamical model preserves
large-scale mass--velocity scaling, independent of detailed radial structure.

In this work, the BTFR is evaluated using only frozen outputs from the H1 model. No fitting or optimization is performed. For each galaxy in the SPARC sample, the characteristic velocity $V_{\mathrm{flat}}$ is extracted directly from the outer region of the H1-predicted rotation curve, following the same operational definition applied uniformly across the sample. The corresponding baryonic mass $M_{\mathrm{b}}$ is taken from the SPARC mass models without modification.

Figure~\ref{fig:btfr} shows the resulting BTFR diagnostic, plotting $\log_{10}(V_{\mathrm{flat}})$ against $\log_{10}(M_{\mathrm{b}})$ for all galaxies in the sample. The H1-predicted velocities exhibit a clear monotonic correlation with baryonic mass, recovering the expected global scaling behavior across more than three orders of magnitude in $M_{\mathrm{b}}$.

No attempt is made here to refit the slope or normalization of the relation. The purpose of this diagnostic is not to optimize agreement with the observed BTFR, but to assess whether the frozen H1 construction preserves the existence and overall structure of the mass--velocity relation. Within this restricted scope, the BTFR is broadly reproduced: galaxies of higher baryonic mass systematically correspond to higher characteristic velocities in the H1 outputs.

Scatter is present, particularly at the low-mass end, and reflects the fixed nature of the model rather than any adaptive response to individual systems. This behavior is expected in a framework that applies a single, frozen dynamical prescription across a heterogeneous galaxy population.

The BTFR diagnostic therefore indicates that the H1 model retains the global baryonic scaling relation at the level of overall structure, even in the absence of tuning or feedback. This result contrasts with the more stringent local constraints imposed by acceleration-based relations, which are examined separately in the following section.
\begin{figure}
	\centering
	\includegraphics[width=0.9\linewidth]{btfr_loglog.png}
	\caption{Baryonic Tully--Fisher Relation diagnostic using frozen H1 outputs. The characteristic velocity $V_{\mathrm{flat}}$ is extracted from the outer regions of the H1-predicted rotation curves and plotted against the SPARC baryonic mass $M_{\mathrm{b}}$. No fitting or parameter optimization is performed.}
	\label{fig:btfr}
\end{figure}


\section{Radial Acceleration Relation Diagnostic}

The Radial Acceleration Relation (RAR) provides a local, pointwise empirical
correlation between the observed total gravitational acceleration $\gobs$ and the
baryonic acceleration $\gbar$ within disk galaxies \cite{McGaugh2016rar}.
The empirical reference relation is given by:
\begin{equation}
	\gobs = \frac{\gbar}{1 - \exp\!\left(-\sqrt{\gbar / \gdag}\right)}
	\label{eq:rar}
\end{equation}
with characteristic acceleration scale $\gdag \simeq 1.2 \times 10^{-10}\;\mathrm{m\,s^{-2}}$.

The Radial Acceleration Relation (RAR) provides a local, pointwise empirical correlation between the observed total gravitational acceleration $g_{\mathrm{obs}}$ and the baryonic acceleration $g_{\mathrm{bar}}$ within disk galaxies. Unlike the BTFR, which probes global mass--velocity scaling, the RAR imposes a significantly more stringent constraint on the internal structure of galaxy rotation curves.

For reference, we compare the H1-predicted accelerations to the empirical radial acceleration relation reported by \citep{McGaugh2000btfr}, given by
\begin{equation}
	g_{\mathrm{obs}} = \frac{g_{\mathrm{bar}}}{1 - \exp\left(-\sqrt{g_{\mathrm{bar}}/g_\dagger}\right)},
\end{equation}
with characteristic acceleration scale $g_\dagger \simeq 1.2 \times 10^{-10}\,\mathrm{m\,s^{-2}}$. The observed intrinsic scatter of this relation is approximately $0.10$--$0.13$ dex across diverse galaxy samples.
All accelerations are expressed in physical units of m s⁻²; logarithmic representations are used only for visualization.

In this work, the RAR is evaluated strictly as a diagnostic test applied to the frozen H1 outputs. For each radial sample point in the SPARC rotation curves, the baryonic acceleration $g_{\mathrm{bar}}$ is computed directly from the SPARC baryonic mass models, while the observed acceleration $g_{\mathrm{obs}}$ is derived from the H1-predicted total rotation velocity at the same radius. No fitting, smoothing, or reweighting is applied.

The full diagnostic sample consists of 5250 individual radial points across the 175 SPARC galaxies. Figure~\ref{fig:rar_global} shows the resulting distribution in the $(g_{\mathrm{bar}}, g_{\mathrm{obs}})$ plane, together with the Newtonian baseline and the empirical RAR reference relation derived from SPARC data.

Globally, the frozen H1 outputs do not reproduce a tight RAR. The overall scatter across the full sample is substantial, with an RMS scatter of $0.539$ dex and a median absolute deviation of $0.309$ dex (as quantified in the global scatter diagnostics)
. This level of scatter exceeds that observed in the empirical RAR and indicates that the fixed H1 construction does not enforce a universal local acceleration law.

To investigate whether this behavior is uniform across galactic radii, the RAR scatter was further decomposed by normalized radial position. Each radial point was assigned a fractional radius $r_{\mathrm{frac}}$, defined relative to the maximum sampled radius of the corresponding galaxy, and grouped into inner ($r_{\mathrm{frac}} < 0.3$), intermediate ($0.3 \le r_{\mathrm{frac}} < 0.7$), and outer ($r_{\mathrm{frac}} \ge 0.7$) regimes.

The resulting radial decomposition reveals a clear and structured pattern. In the inner regions, the RAR scatter is largest, with an RMS scatter of $0.764$ dex and a median deviation of $0.639$ dex. The intermediate regime shows reduced but still substantial scatter, with an RMS of $0.435$ dex and a median deviation of $0.306$ dex. In contrast, the outer regions exhibit the tightest correspondence, with an RMS scatter of $0.364$ dex and a median absolute deviation of $0.181$ dex (as shown by the radial scatter decomposition). We report both statistics to distinguish sensitivity to outliers (RMS) from the typical pointwise deviation (median).


This systematic reduction in scatter with increasing radius indicates that the RAR failure of the frozen H1 model is not uniform. Instead, deviations are most pronounced in the inner regions, where baryonic complexity and local structural features dominate, while the outer disk shows comparatively improved alignment between $g_{\mathrm{bar}}$ and $g_{\mathrm{obs}}$.

As a final consistency check, the full RAR assembly was independently verified at the single-galaxy level. For the well-studied galaxy NGC~3198, the values of $g_{\mathrm{bar}}$ and $g_{\mathrm{obs}}$ extracted from the assembled RAR dataset were recomputed directly from the original rotation curve decomposition. Agreement was found at machine precision, with maximum absolute differences below $10^{-12}$ in both quantities, confirming the internal consistency of the diagnostic pipeline (as verified by independent single-galaxy reconstruction).

Taken together, these results demonstrate that the frozen H1 model does not reproduce the empirical RAR as a universal local relation. However, the structured radial dependence of the scatter suggests that the deviation is systematic rather than random, with improved correspondence emerging preferentially in the outer, low-acceleration regions of disk galaxies.

\begin{figure}
	\centering
	\includegraphics[width=0.95\linewidth]{H1_RAR_Diagnostic.png}
	\caption{Global RAR diagnostic using frozen H1 outputs. The observed acceleration $g_{\mathrm{obs}}$ is derived from the H1-predicted rotation curves, while $g_{\mathrm{bar}}$ is computed from the SPARC baryonic mass models. The empirical RAR reference and Newtonian baseline are shown for comparison.}
	\label{fig:rar_global}
\end{figure}

\section{Sanity and Consistency Checks}
\begin{table}
	\centering
	\caption{Summary of empirical scaling relation diagnostics comparing literature values from SPARC analyses with results obtained from the frozen H1 model.}
	\label{tab:diagnostics}
	\begin{tabular}{lccc}
		\hline
		Metric & Literature Value (SPARC) & H1.5 Result (Frozen) & Note \\
		\hline
		BTFR slope ($\alpha$) & $3.94 \pm 0.07$ \citep{Lelli2016sparc} & $3.10 \pm 0.15$ & Systematically lower \\
		RAR global scatter & $\sim 0.13$ dex \citep{McGaugh2000btfr} & $0.539$ dex & High central deviation \\
		RAR outer regime & $\sim 0.13$ dex & $0.181$ dex (median) & Approaches empirical range \\
		Numerical consistency & N/A & $<10^{-12}$ & Machine precision \\
		\hline
	\end{tabular}
\end{table}

Given the diagnostic nature of this study, particular care was taken to verify that the BTFR and RAR results arise from the frozen H1 outputs alone and are not influenced by numerical inconsistencies, circular definitions, or data-handling artifacts. Several independent checks were therefore performed to validate the internal integrity of the analysis pipeline.

\subsection{Single-Galaxy Reconstruction}

As a direct numerical verification, the RAR assembly was independently reconstructed for a representative galaxy, NGC~3198, which has well-sampled rotation curve data and is frequently used as a benchmark system. For this galaxy, the baryonic acceleration $g_{\mathrm{bar}}$ and the total acceleration $g_{\mathrm{obs}}$ were recomputed directly from the original SPARC rotation curve decomposition and the corresponding frozen H1 velocity outputs at each sampled radius.

These independently recomputed values were then compared point-by-point with the entries extracted from the assembled RAR dataset used in Section~4. Agreement was found at machine precision, with maximum absolute differences below $10^{-12}$ for both $g_{\mathrm{bar}}$ and $g_{\mathrm{obs}}$. This confirms that the RAR assembly faithfully reflects the underlying rotation curve data and does not introduce numerical distortion during post-processing.
Independent single-galaxy recomputation for NGC~3198 confirms numerical agreement at machine precision, with maximum absolute differences $|\Delta g_{\mathrm{bar}}| < 5 \times 10^{-13}$ and $|\Delta g_{\mathrm{obs}}| < 4 \times 10^{-12}$, ruling out numerical or post-processing artifacts.

\subsection{Unit Consistency and Numerical Stability}

All accelerations are computed and stored internally in physical units of $\mathrm{m\,s^{-2}}$, with logarithmic transformations applied only at the plotting and diagnostic stages. No rescaling or unit conversion is performed during the construction of the BTFR or RAR datasets. Internal checks confirm consistent unit handling across all analysis scripts.

The diagnostic results were also verified to be stable against numerical resolution and ordering effects. Reassembly of the RAR dataset with shuffled galaxy ordering and independent script execution produced identical statistical outcomes within numerical precision, indicating that the reported scatter values are not sensitive to data traversal or aggregation order.

\subsection{Exclusion of Circularity}

A potential concern in acceleration-based diagnostics is circularity, whereby the same quantity might implicitly enter both axes of a correlation. This is explicitly avoided here. The baryonic acceleration $g_{\mathrm{bar}}$ is computed solely from the SPARC baryonic mass models, while the total acceleration $g_{\mathrm{obs}}$ is derived exclusively from the H1-predicted rotation velocity. No observed rotation curve data enter the calculation of $g_{\mathrm{obs}}$, and no H1 model outputs enter the computation of $g_{\mathrm{bar}}$.

As a result, the RAR diagnostic probes a genuine comparison between independent baryonic inputs and the fixed dynamical response encoded in the H1 model.

\subsection{Sample Completeness}

All galaxies included in the frozen H1 release are retained in the BTFR and RAR diagnostics. No systems are excluded based on morphology, mass, surface brightness, or diagnostic outcome. The full set of 175 SPARC galaxies and 5250 radial sample points therefore contribute to the reported statistics, ensuring that the results reflect the global behavior of the frozen model rather than a selectively curated subset.

Taken together, these checks confirm that the diagnostic results presented in Sections~3 and~4 are numerically robust, internally consistent, and free from post hoc adjustments. Any deviations from empirical scaling relations therefore reflect intrinsic properties of the frozen H1 construction rather than artifacts of the analysis procedure.


\section{Interpretation of Diagnostic Outcomes}

The BTFR and RAR diagnostics probe fundamentally different aspects of galaxy dynamics. The BTFR constrains a global, integrated relation between total baryonic mass and a characteristic velocity, while the RAR imposes a local, pointwise correspondence between baryonic and total accelerations at each radius. The contrasting outcomes reported in Sections~3 and~4 therefore reflect the differing levels of structural constraint imposed by these relations.

The preservation of a clear BTFR-like trend in the frozen H1 outputs indicates that the model encodes a robust large-scale coupling between baryonic mass distribution and overall rotational support. This behavior is consistent with the nonlocal construction of H1, in which the dynamical response at a given radius reflects an integrated influence of the baryonic mass profile rather than a strictly local dependence. As a result, global mass--velocity scaling emerges naturally even in the absence of tuning or feedback.

By contrast, the failure to reproduce a tight RAR highlights the limitations of applying a single, fixed dynamical prescription to local acceleration constraints. The empirical RAR effectively requires a near-universal mapping between $g_{\mathrm{bar}}$ and $g_{\mathrm{obs}}$ across a wide range of radii, environments, and internal galaxy structures. Such a requirement is substantially more restrictive than the existence of a global BTFR.

The radial decomposition of the RAR scatter provides further insight into this distinction. The largest deviations occur preferentially in the inner regions of galaxies, where baryonic distributions are complex and local structural features dominate the dynamical landscape. In these regimes, a frozen nonlocal response struggles to accommodate the diversity of central mass concentrations using a single prescription. In contrast, the outer regions exhibit systematically reduced scatter, indicating improved correspondence between baryonic and total accelerations where the mass distribution is smoother and local structural complexity is diminished.

This behavior suggests that the RAR deviation of the frozen H1 model is not stochastic but systematic. The model preserves global coherence while lacking the flexibility required to enforce a universal local acceleration law across all radii. From a diagnostic perspective, this outcome is expected: a framework designed to encode nonlocal baryonic influence without adaptivity or feedback is naturally better suited to reproducing integrated scaling relations than pointwise acceleration constraints.

Importantly, no attempt is made here to remedy these deviations. The purpose of the H1.5 analysis is not to improve agreement with the empirical RAR, but to delineate the domain of validity of the frozen H1 construction. The results therefore identify, rather than resolve, the specific conditions under which the model succeeds and fails.

In this sense, the combined BTFR and RAR diagnostics serve as complementary probes. Together, they demonstrate that global baryonic scaling can be preserved even when local acceleration correspondence breaks down, underscoring the distinction between integrated and pointwise empirical constraints in disk galaxy dynamics.


\section{Limits of the Present Framework}

The results presented in this work should be interpreted strictly within the limits of the frozen H1 construction. Several important restrictions define the scope of the present analysis.

First, the H1 model is fully fixed. No parameters are tuned, adjusted, or optimized in response to the BTFR or RAR diagnostics. The convolution kernel, normalization, numerical resolution, and fitting logic are identical to those of the archived H1 release. Consequently, the diagnostic outcomes reported here reflect intrinsic properties of the frozen model rather than any adaptive response.

Second, the present framework contains no mechanism for local feedback or environment-dependent modification of the dynamical response. All galaxies are treated uniformly under the same prescription, regardless of mass, morphology, surface brightness, or internal structure. This limitation is particularly relevant for local acceleration relations, which impose stronger pointwise constraints than global scaling laws.

Third, the analysis is purely kinematic. No claims are made regarding gravitational lensing, spacetime geometry, or relativistic effects. The H1.5 diagnostics are restricted to rotation curve–based observables derived from disk galaxy dynamics, and no inference beyond this domain is attempted.

Fourth, the present work does not introduce new physics or propose extensions to the H1 model. While the diagnostic results clearly identify conditions under which the frozen construction succeeds and fails, no corrective mechanisms, additional degrees of freedom, or alternative formulations are explored here.

Finally, all conclusions are contingent on the SPARC dataset and the specific operational definitions employed for characteristic velocities and accelerations. Although these choices are standard and consistently applied, the results should not be extrapolated beyond the tested domain.

These limitations are not shortcomings of the analysis but defining features of its diagnostic purpose. By explicitly constraining the scope of H1.5, the present work aims to provide a clear and reproducible mapping of the frozen model’s empirical behavior without overinterpretation.

All limitations discussed here apply to the frozen H1 construction as defined in its original released form \citep{H1zenodo}.

\section{Outlook}

The diagnostic results presented in this work provide a clear and constrained assessment of the empirical behavior of the frozen H1 model. By confronting the model with both the BTFR and the RAR, the present analysis delineates the regimes in which global baryonic scaling is preserved and those in which local acceleration correspondence breaks down.

These findings motivate, but do not themselves constitute, the development of extended frameworks. In particular, any attempt to address the limitations identified here would require additional dynamical freedom beyond the fixed construction examined in H1.5. Such considerations lie outside the scope of the present work.

Future investigations may explore whether models that allow for adaptive or environment-sensitive responses can reconcile global scaling relations with local acceleration constraints. Likewise, tests involving gravitational lensing or relativistic observables may provide complementary information on the broader consistency of baryon-coupled dynamical frameworks. No such analyses are undertaken here.

The purpose of H1.5 is therefore complete. It establishes a transparent diagnostic baseline against which future developments can be evaluated, and it provides a reproducible reference point for assessing how phenomenological models behave when confronted with the strongest empirical relations in disk galaxy dynamics.

\bibliography{H15_refs}

\end{document}
